\documentclass[a4paper, 12pt]{article}
\usepackage[english]{babel}
\usepackage[utf8]{inputenc}
\usepackage[T1]{fontenc}
\usepackage{lmodern}
\usepackage{hyperref}
\usepackage[numbers, sort&compress]{natbib}
\usepackage{calc}
\usepackage{fancyhdr}
\usepackage{graphics}
\usepackage{nowidow}
\usepackage{color}
\usepackage{subcaption}


\newlength{\eqMargin}
\newlength{\eqHorizMargin}
\newlength{\eqVertMargin}

\setlength{\eqMargin}{20mm}
\setlength{\eqHorizMargin}{\eqMargin}
\setlength{\eqVertMargin}{\eqMargin}

% Paper
\setlength{\paperwidth}{210mm}
\setlength{\paperheight}{297mm}

% Rid the extra space
\setlength{\hoffset}{-1in}
\setlength{\voffset}{-1in}
\addtolength{\hoffset}{\eqHorizMargin}
\addtolength{\voffset}{\eqVertMargin}

% Set margin from the page border (horizontal)
\setlength{\oddsidemargin}{0pt}
\setlength{\evensidemargin}{0pt}

% Header
\setlength{\topmargin}{0pt}
\setlength{\headheight}{30pt}
\setlength{\headsep}{18pt}
\renewcommand{\headrulewidth}{0pt}

% Footer
\addtolength{\footskip}{18pt}
\renewcommand{\footrulewidth}{0pt}

% Margin notes
\setlength{\marginparsep}{0pt}
\setlength{\marginparwidth}{0pt}

% Text
\setlength{\textwidth}{\paperwidth - \hoffset - \hoffset - 25.4mm - 25.4mm}
\setlength{\textheight}{\paperheight - \voffset - \topmargin - \headheight - \headsep - \footskip - \voffset - 25.4mm - 25.4mm}

%\setlength{\labelwidth}{20mm}

% Hyperref settings
\hypersetup{
    unicode=true,					% non-Latin characters in Acrobat's bookmarks
    pdftoolbar=true,				% show Acrobat's toolbar?
    pdfmenubar=true,				% show Acrobat's menu?
    pdffitwindow=false,				% window fit to page when opened
    pdfstartview={FitH},			% fits the width of the page to the window
    pdftitle={S-26.3120 Radio Engineering, laboratory course},	% title
    pdfauthor={Tuomas Leinonen} {Sampo Salo},	% author
    pdfsubject={Radio Engineering},	% subject of the document
    pdfcreator={LaTeX},				% creator of the document
    pdfproducer={Aalto},			% producer of the document
    pdfkeywords={radio} {gsm} {bs} {tx},	% list of keywords
    pdfnewwindow=true,				% links in new window
    colorlinks=true,				% false: boxed links; true: colored links
    linkcolor=black,				% color of internal links
    citecolor=black,				% color of links to bibliography
    filecolor=black,				% color of file links
    urlcolor=black					% color of external links
}

% Bad hyphenation
%\hyphenation{}

\definecolor{dkred}{rgb}{0.6, 0, 0}
\definecolor{dkgrn}{rgb}{0, 0.6, 0}
\definecolor{dkblue}{rgb}{0, 0, 0.6}

\pagestyle{fancy}
\lhead{S-26.3120 Radio Engineering, laboratory course\\Lab 1: GSM BS TX -- Pre-study}
\rhead{Sampo Salo, 79543L\\Tuomas Leinonen, 84695P}
\cfoot{\thepage}


\begin{document}

\begin{titlepage}
\pagestyle{empty}
\begin{center}

\vspace*{3cm}
\noindent\LARGE{\textbf{S-26.3120 Radio Engineering, laboratory course}}

\vspace*{2cm}

\Large{\textbf{Lab 1: GSM Basestation Transmitter}}\\

\vspace*{1.5cm}

\large{\textbf{Pre-study report}}\\
\vspace{1.5cm}
\large{\today}
	
\vspace*{3cm}
\large{
	\begin{tabular}{l l}
		\textbf{Group 4:} 	& \\
		Sampo Salo			& 79543L	\\
		Tuomas Leinonen 	& 84695P		
	\end{tabular}
}

\end{center}

\end{titlepage}

\section*{Measurement and setup descriptions}

\textit{Give a short description how the measurements will be carried out, 
how measurement and test equipment setup is connected during measurements.}


\subsection*{Measurement 1: Dynamic Downlink Power Control Level}

In the first measurement, the transmitter power is measured using five 
different power control levels. The final measurement setup will made 
using the following connections: TX $\rightarrow$ Cable $\rightarrow$ 
Attenuator $\rightarrow$ Cable $\rightarrow$ Spectrum Analyzer. Thus 
before we begin with the actual measurement, the combined effect of 
the cables and attenuator need to be determined. This may be carried 
out using a VNA or a power meter (both of which need to be calibrated, 
just like the spectrum analyzer for the actual measurement).

For the actual measurement we focus a single frequency from inside the 
BS TX band of $935 - 960$ MHz. As this frequency was not specified in 
the instructions, the center frequency (947,5 MHz) would be our guess.
This will most likely be set in the BS software. We will ask the assistant. 
The spectrum analyzer will be configured as follows: given center 
frequency (see above), zero span, video and resolution bandwidth of 
30 kHz and 500-measurements averaging.


\subsection*{Measurement 2: In-band output spectrum}

In the second measurement the setup will be the same as before, but 
the measurement is done using only one transmitted power level chosen 
based on the first measurement (the measurements should thus be ordered).
Maximum power level is suggested in the instructions.

Here we are interested of the shape of the spectrum, not of the absolute 
power. Thus we will measure the relative power levels around the carrier 
with specified offsets. These offsets are given in the GSM specifications, 
to which the measured levels are eventually compared.


\subsection*{Measurement 3: Interference}

Yet again the measurement setup will be same as before. In contrast to the 
measurements made before, here we are interested of a much larger bands than 
previously. The interference will be worst when the TX power is at its maximum, 
and at harmonic multiples of the carrier, as suggested by the instructions. 
Thus we will focus our efferts on those frequencies as the measurement is 
basically ``a peak search''.


\subsection*{Measurement 4: Diplexer characterization}

The last measurement will be done using a Vector Network Analyzer, which is 
first calibrated (and verified) using a VNA calibration set for the intented 
measurement setup: cables, frequency band (e.g., $850 - 1000$ MHz) and other 
related settings (enough points per sweep, suitable averaging factor). After 
this, the VNA is connected successively to the port pairs (VNA $\leftrightarrow$ 
``port pair'' $\leftrightarrow$ VNA), and the $S_{21}$-parameter 
will be measured (only the absolute value is required for insertion loss). 
This raw data is stored onto a USB-memory for postprocessing.


\section*{Problem 1}

\textit{What is the minimum TX power needed for a 1 km mobile downlink assuming an ideal 
half-wavelength dipole on the base station (TX) and an ideal monopole on ground 
plane at the mobile end (RX), the receiver sensitivity is $-95$ dBm, SNR = 15 dB.}

We assume ideal, direct line-of-sight link in totally empty space between the base 
station and the mobile receiver, neglecting nonideal effects such as multipath, 
shadowing and weather. Solving the Friis transmission equation \cite{pozar}

\begin{equation}
P_\mathrm{R} = SNR \times P_\mathrm{min} = G_\mathrm{T} \times G_\mathrm{R} \times \left(\frac{\lambda}{4 \pi r}\right)^2 \times P_\mathrm{T} 
\end{equation}

\noindent for transmission power $P_\mathrm{T}$, using the following values

\begin{table}[!h]
\begin{tabular}{llrll}
Transmitter power 		& $P_\mathrm{T}$ 		& ??? 						& 			& \\
Receiver sensitivity 	& $P_\mathrm{min}$		& $-95$ 					& dBm 		& \\
Signal-to-noise ratio 	& $SNR$ 				& $15$ 						& dB 		& \\
Link length		 		& $r$ 					& $1000$ 					& m 		& \\
Frequency				& $f$ 					& $947,\!5$ 				& MHz 		& Taken as the center frequency. \\
TX antenna gain			& $P_\mathrm{T}$ 		& $2,\!16$ 					& dBi 		& $\lambda/2$-dipole \cite{gains} \\
RX antenna gain	 		& $G_\mathrm{R}$		& $5,\!16$ 					& dBi 		& $\lambda/4$-monopole, infinite ground plane \cite{gains} \\
The speed of light		& $c$ 					& $2,\!998 \times 10^8$ 	& m/s 		& $\lambda = c / f$\\
\end{tabular}
\end{table}

\noindent while also substituting $\lambda = c / f$, yields

\begin{equation}
P_\mathrm{T} = \frac{SNR \times P_\mathrm{min}}{G_\mathrm{T} \times G_\mathrm{R} \times \displaystyle \left(\frac{c}{4 \pi f r}\right)^2}.
\end{equation}

\noindent By taking $10 \lg \left( \; \right)$ dB on both sides, we have an equation with 
decibels (and its derivatives) from which we finally solve $P_\mathrm{T}$:

\begin{eqnarray}
P_\mathrm{T,\,dBm} & = & SNR_\mathrm{dB} + P_\mathrm{min,\,dBm} - G_\mathrm{T,\,dBi} - G_\mathrm{R,\,dBi} - 20 \lg \left(\frac{c}{4 \pi f r}\right) \mathrm{\;dB}\\[2pt]
& = & \left[ 15 - 95 - 2,\!16 - 5,\!16 - 20 \lg \left(\frac{2,\!998 \times 10^8}{4 \pi \times 947,\!5 \times 10^8 \times 1000}\right) \right] \mathrm{\;dBm}\\[2pt]
& = & +4,\!6591... \mathrm{\;dBm} \approx +4,\!66 \mathrm{\;dBm} \\[2pt]
& = & 2,\!9235... \textrm{ mW} \approx 2,\!92 \textrm{ mW}
\end{eqnarray}

\noindent The minimum transmitter power is thus $+4,\!66 \mathrm{\;dBm}$ or $2,\!92 \textrm{ mW}$.


\section*{Problem 2}

\textit{Compare the I/Q-diagrams of the MSK- and QPSK-modulation methods to each other. 
Which fundamental difference do they have?}

\begin{figure}[!h]
	\begin{center}
	\begin{subfigure}[c]{0.3\textwidth}
		\includegraphics{img/MSK_IQ.pdf}
	\end{subfigure}
	\begin{subfigure}[c]{0.3\textwidth}
		\includegraphics{img/QPSK_IQ.pdf}
	\end{subfigure}
	\caption{I/Q constellation diagrams of the FSK (left) and QSPK (right) modulations. \cite{parts}}
	\end{center}
\end{figure}
		
		
		
MSK, or Minimum Shift Keying, refers to a specific type of Frequency Shift Keying (FSK), where the 
information is coded into the frequency of the signal. Simple two-state FSK modulator would be 
achieved by controlling a PLL with the digital base band signal. In MSK this control signal is 
low-pass filtered to achieve smooth transitions to between two frequencies (i.e., states). In 
GMSK, a MSK variant, the signal is gaussian filtered.

QPSK, or Quadrature Phase Shift Keying, refers to a four-state a phase modulation. Now quadrature 
modulation is required. In the basic form of QPSK, the transitions are not controlled; they may 
be abrupt and they may include unwanted zero-crossings. There are more advanced alternatives with 
better amplitude-envelope, such as $\pi/4$-QPSK or Offset-QPSK.

The I/Q diagrams of the previous modulation schemes are presented in the previous figure. Both 
of the methods store no information in their amplitude and are thus power efficient and provide 
adequate protection against fading. MSK is better in terms of power efficiency and spurious 
emissions than the QPSK, but the QPSK is the most noise tolerant of the two. \cite{parts}

For a more thorough discussion, see \cite{parts}.


\section*{Problem 3}

\textit{What is the function of the diplexer between the antenna and the RF 
circuitry and what are the diplexer’s most important performance parameters? 
What is the difference between a diplexer and a circulator?}

In FDD systems, when both a transmitter and a receiver are part of a single unit, 
there's inherent possibility of coupling between the TX and RX, be it due to shared 
antenna or due to coupling between two separate antennas, for example. Without 
good enough isolation leakinghigh-power TX may desensitize, saturate and/or 
possibly damage the receiver circuitry. Thus we need to isolate the RX from 
the TX.

When using a single antenna, required isolation in theory is best achieved 
with a circulator. In the real world, though, practical circulators are hard 
to come by. And those antenna mismatches should be accounted as well. Thus 
the engineers have resorted to diplexers to achieve the required isolation. 
Where as circulators are wide-band, diplexers are narrowband due to their 
filter-like nature. 

Diplexers are in essence a power divider/combiner and two neighbouring, but 
not overlapping, high-quality bandpass filters with steep roll-offs packed 
in a single 3-port. Thus RX-TX isolation is achieved with two filter stop bands. 
In addition to the isolation they work as a spurious emissions filter and a 
pre-select filter on the TX and RX sides, respectively.

Thus multiple diplexers are required for multi-band operation. They are 
characterized, by the TX-RX isolation and in both bands by the corner
frequencies, in-band insertion losses and stop-band attenuations. Other 
practical characteristics include (but are not limited to) size, cost, 
temperature stability, maximum power, tunability (for SDRs) and linearity. 
In mobile handsets, diplexers are most commonly realized using acoustic 
wave technologies. \cite{kandi}

An example of a commercially available SAW dupler for cellular applications 
is available here: \cite{triquint}.


\section*{Problem 4}

\textit{What is the minimum isolation required from the diplexer between TX and RX 
if we want to avoid saturation of the RX pre-amp (LNA) by the leakage TX (blocker) 
signal, assuming $P_\mathrm{TX,\,max} = 10$ W, and the maximum RX pre-amp (LNA) 
input power level without desensitisation (or distortion) to be $-55$ dBm.}

The maximum TX power leaking to RX satisfies

\begin{equation}
P_\mathrm{RX,\,max} = \frac{P_\mathrm{TX,\,max}}{I_\mathrm{PDX,\,min}},
\end{equation}
\vspace{2pt}

\noindent from which we can find the required minimum diplexer isolation $I_\mathrm{DPX,\,min}$. 
Converting to decibels we obtain ($P_\mathrm{TX,\,max} = 40$ dBm)

\begin{equation}
I_\mathrm{DPX,\,min,\,dB} = P_\mathrm{TX,\,max,\,dBm} - P_\mathrm{RX,\,max,\,dBm} 
	= \left[ 55 - (-40) \right] \mathrm{\;dB}
	= 95 \mathrm{\;dB}
\end{equation}
\vspace{2pt}

\noindent Thus the diplexer TX-RX isolation should be at least $95 \mathrm{\;dB}$ or $3,\!16 \times 10^9$.


\begin{thebibliography}{9}%\itemsep 7pt\parskip -5pt 


\bibitem{pozar} D.\ M.\ Pozar, 
	\textit{Microwave Engineering}, 
	J.\ Wiley \& Sons, 4th Edition, 2012. 
	ISBN: 978-0-470-63155-3.
	
\bibitem{gains} J.\ C.\ Logan, J.\ W.\ Rockway, 
	``Dipole and Monopole Antenna Gain and Effective Area for Communication Formulas.''
	Available online at \url{www.dtic.mil/cgi-bin/GetTRDoc?AD=ADA332891}
	[Retrieved: November 25, 2013].

\bibitem{parts} M.\ Steer, 
	\textit{Microwave and RF Design -- A Systems Approach}, 
	SciTech Publishing, 2010. 
	ISBN: 978-1-891-12188-3.
	
\bibitem{kandi} T.\ Leinonen, ``Antenna and front-end challenges for mobile software-defined radio receiver,''
	B.\ Sc.\ (Tech.) thesis (in Finnish), 2012. Available online at \url{http://urn.fi/URN:NBN:fi:aalto-201301161154}.
	
\bibitem{triquint} Triquint 856908 (836.5/881.5 MHz SAW Duplexer) datasheet. Available online at 
\url{http://triquint.com/products/d/DOC-A-00000661} [Retrieved: December 1, 2013].
	
%\bibitem{iet} R.\ J.\ Collier, A.\ D.\ Skinner (editors), \textit{Microwave Measurements}, 
%	The Institution of Engineering and Technology, 3rd Edition, 2007. ISBN: 978-0-86341-735-1.

\end{thebibliography}

\end{document}
